\documentclass[12pt, letterpaper, twoside]{article}
\usepackage[T2A]{fontenc}
\usepackage[utf8]{inputenc}
\usepackage[russian,english]{babel}
\usepackage{amsmath,amssymb}
\usepackage{indentfirst}
\usepackage{hyperref}

\title{Описательная часть расчетной работы №2 по математической статистике за 6 триместр}
\author{Вихляев Егор, ММТ-2}
\date\today

\begin{document}
	
	\maketitle
	
	\section{Задание №1}
	Для выбранного набора данных построить корреляционное поле.\\
	
	Для построения корреляционного поля, определимся с исходными данными. В левом столбце имеем независимые переменные $x_i$, в правом столбце -- зависимые переменные $y_i$. Строим поле по двум представленным столбцам через точечную диаграмму в Excel (все вычисления в расчетном файле). 
	\\
	Там же добавим и уравнение детерминированной части регрессии (в Excel оно называется «линия тренда»): $$Y = 2.0814 \cdot X + 8.8277.$$ Обозначим график уравнения красной штрихованной линией.
	
	\section{Задание №2}
	Построить коэффициенты корреляции Пирсона и Спирмена. Проверить значимость
	коэффициентов при уровне значимости $\alpha = 0.1$. Сделать выводы о наличии
	корреляционной зависимости между переменными и характере их
	корреляционной зависимости, исходя из вычисленных значений
	коэффициентов. \\
	
	\begin{enumerate}
		\item Построить коэффициенты корреляции Пирсона и Спирмена.\\
		Коэффициент Пирсона считается по следующей формуле и равен: $$r_{xy}^* = \frac{\overline{x \cdot y}-\overline{x} \cdot \overline{y}}{\sqrt{S_x^2 \cdot S_y^2}} \approx 0.531.$$
		В Excel его можно вычислить через функцию =$\text{ПИРСОН(X;Y)}$.\\
		Ранговый коэффициент корреляции Спирмена считается по следующей формуле и равен: $$r_s = 1 - \frac{6}{n \cdot (n^2 - 1)} \cdot \sum_{i=1}^n (rang(X_i) - rang (Y_i))^2 \approx 0.339.$$
		В Excel его можно вычислить либо по формуле, приведенной выше, расписав ее элементы по столбцам, либо найти ранги элементов, а затем применить к ним функцию =КОРРЕЛ(диап. рангов X; диап. рангов Y). Для ясности процесса, в расчетной работе мы привели первый вариант решения.
		\item Проверить значимость коэффициентов при уровне значимости $\alpha = 0.1$.\\
		\begin{enumerate}
			\item Начнем с коэффициента Пирсона.\\
			Выдвинем следующую нулевую гипотезу: $$H_0: r_{xy}^* = 0.$$
			Для проверки значимости коэффициента Пирсона, потребуется сравнить по модулю две статистики, а именно: 
			$$t_{\text{набл}} = \frac{\sqrt{n-2} \cdot r_{xy}^*}{\sqrt{1-(r_{xy}^*)^2}} = \frac{\sqrt{50-2} \cdot 0.531}{\sqrt{1-{0.531}^2}} \approx 4.342,$$ 
			$$t_{\text{кр}} = x_{1-\frac{\alpha}{2}}[St_{n-2}] = x_{0.95}[St_{48}] = 1.677.$$
			Исходя из вычисленных статистик, имеем следующее неравенство: 
			$$|t_{\text{набл}}| > t_{\text{кр}} \Rightarrow$$
			$\Rightarrow H_0$ -- не принимается $\Rightarrow$ коэффициент Пирсона $r_{xy}^*$ -- значим! 
			\item Проверим на значимость ранговый коэффициент Спирмена.\\
			$$H_0: r_s = 0.$$
			Находим необходимые статистики: 
			$$t_{\text{набл}} = \frac{\sqrt{n-2} \cdot r_s}{\sqrt{1-r_s^2}} = \frac{\sqrt{50-2} \cdot 0.339}{\sqrt{1-{0.339}^2}} \approx 2.496,$$ 
			$$t_{\text{кр}} = x_{1-\frac{\alpha}{2}}[St_{n-2}] = x_{0.95}[St_{48}] = 1.677.$$
			Исходя из вычисленных статистик, имеем следующее неравенство: 
			$$|t_{\text{набл}}| > t_{\text{кр}} \Rightarrow$$
			$\Rightarrow H_0$ -- не принимается $\Rightarrow$ ранговый коэффициент Спирмена $r_s$ -- значим! 
		\end{enumerate}
	\item Сделать выводы о наличии корреляционной зависимости между переменными и характере их корреляционной зависимости, исходя из вычисленных значений коэффициентов.\\
	Исходя из вычисленных значений коэффициентов, можно сделать следующие выводы:
	\begin{enumerate}
		\item Согласно коэффициенту Пирсона $0.3 <r_{xy}^* \approx 0.531 < 0.7$, переменные $X_i$ и $Y_i$ линейно зависимы, связь средней тесноты.
		\item Согласно ранговому коэффициенту Спирмена $0.3 <r_s* \approx 0.339 < 0.7$, переменные $X_i$ и $Y_i$ линейно зависимы, связь средней тесноты.
	\end{enumerate}
	Как следствие слабой зависимости, мы не можем явно сказать, что имеется прямая («чем больше, тем больше») или обратная («чем больше, тем меньше») зависимости. Это потверждает и корреляционное поле из задания №1.
	\end{enumerate}

\section{Задание №3}
Записать линейную регрессионную модель. Выписать оценки неизвестных параметров модели.\\ \\

Линейная регрессионная модель имеет вид: $$Y = a + b \cdot X + \epsilon, $$ где $Y$ -- отклик (зависимая переменная), $X$ -- фактор (независимая переменная), $\epsilon$ -- случайная компонента (суммарная ошибка). Из задания №1 мы знаем, что наша детерминированная часть регрессионой модели имеет вид: $$Y = 2.0814 \cdot X + 8.8277.$$

Сделаем оценки $\hat{a}$ и $\hat{b}$ неизвестных параметров $a$ и $b$ с помощью метода наименьших квадратов (МНК), используя заведомо выведенные формулы:
$$\hat{a} = \overline{Y} - \hat{b} \cdot \overline{X} = 9.33 - 2.0398*0.24 \approx 8.8376,$$
$$\hat{b} = \frac{\overline{X \cdot Y} - \overline{X} \cdot \overline{Y}}{S_x^2} = \frac{12.177-0.24*9.33}{74.99} \approx 2.0398.$$

Итак, оценки неизвестных параметров $a$ и $b$: $\hat{a} \approx 8.8376$, $\hat{b} \approx 2.0398$. Как можем наблюдать, наши оценки неизвестных параметров примерно совпадают со значениями неизвестных параметров, выведенных с помощью Excel в задании №1: $a = 2.0814, b = 8.8277$.

\section{Задание №4}
Для нелинейных моделей $y= a+b \cdot x^2 + \epsilon, y = a + \frac{b}{x} + \epsilon$ найти МНК-оценки коэффициентов и коэффициент детерминации. Выбрать лучшую из нелинейных моделей и выписать ее. Сравнить выбранную нелинейную модель с линейной моделью
\begin{enumerate}
	\item Найдем МНК-оценки для первой модели $y= a+b \cdot x^2 + \epsilon.$ 
	$$W = \sum_{i=1}^n(y_i- a - b \cdot x_i^2)^2 \rightarrow min.$$
	$$\frac{\partial W}{\partial a} = \sum_{i=1}^n 2 \cdot (y_i- a - b \cdot x_i^2) \cdot (-1) = 0,$$
	$$\frac{\partial W}{\partial b} = \sum_{i=1}^n 2 \cdot (y_i- a - b \cdot x_i^2) \cdot (-x_i^2) = 0$$
	
	\begin{equation*}
		\begin{cases}
			\sum_{i=1}^n y_i - a \cdot n - b \sum_{i=1}^n x_i^2 = 0 \ /:n \\
			\sum_{i=1}^n y_i \cdot x_i^2 - a\sum_{i=1}^n x_i^2 - b \sum_{i=1}^n x_i^4 = 0 \ /:n
		\end{cases}
	\end{equation*}

	\begin{equation*}
		\begin{cases}
			\overline{y_i} - a - b \cdot \overline{x^2} = 0 \\
			\overline{x^2 \cdot y} - a\cdot \overline{x^2} - b \cdot \overline{x^4} = 0
		\end{cases} \Rightarrow 
	\end{equation*}

	\begin{equation*}
		 \Rightarrow 
		\begin{cases}
			\hat{a} = \overline{y} - b \cdot \overline{x^2}, \\
			\hat{b} =  \frac{\overline{x^2 \cdot y} - \overline{y} \cdot \overline{x^2}}{\overline{x^4} -(\overline{x^2})^2}
		\end{cases} \Rightarrow
	\end{equation*}

	\begin{equation*}
		\Rightarrow 
		\begin{cases}
			\hat{a} =  2.6562,\\
			\hat{b} = 1.3804
		\end{cases} 
	\end{equation*}
	Коэффициент детерминации $$R_1^2 = 0.793.$$
	
	\item Найдем МНК-оценки для второй модели $y = a + \frac{b}{x} + \epsilon.$
	$$W = \sum_{i=1}^n(y_i- a - \frac{b}{x_i})^2 \rightarrow min.$$
	$$\frac{\partial W}{\partial a} =\sum_{i=1}^n 2 \cdot (y_i- a - \frac{b}{x_i}) \cdot (-1) = 0, /: (-2)$$
	$$\frac{\partial W}{\partial b} = \sum_{i=1}^n 2 \cdot (y_i- a - \frac{b}{x_i}) \cdot (-\frac{1}{x_i}) = 0 /: (-2)$$
	
	\begin{equation*}
		\begin{cases}
			\sum_{i=1}^n y_i - a \cdot n - b \sum_{i=1}^n \frac{1}{x_i} = 0 \ /:n \\
			\sum_{i=1}^n \frac{y_i}{x_i}- a\sum_{i=1}^n \frac{1}{x_i} - b \sum_{i=1}^n \frac{1}{x_i^2} = 0 \ /:n
		\end{cases}
	\end{equation*}

	\begin{equation*}
		\begin{cases}
			\overline{y} - a - b \cdot \overline{(\frac{1}{x})} = 0 \\
			\overline{(\frac{y}{x})} - a\cdot \overline{(\frac{1}{x})} - b \cdot \overline{(\frac{1}{x^2})} = 0
		\end{cases} \Rightarrow 
	\end{equation*}

	\begin{equation*}
		\Rightarrow 
		\begin{cases}
			\hat{a} = \overline{y} - b \cdot \overline{(\frac{1}{x})}, \\
			\hat{b} =  \frac{\overline{\frac{y}{x}} - \overline{\frac{1}{x}} \cdot \overline{y}}{\overline{\frac{1}{x^2}} -(\overline{\frac{1}{x}})^2}
		\end{cases} \Rightarrow
	\end{equation*}

	\begin{equation*}
		\Rightarrow 
		\begin{cases}
			\hat{a} =  9.3479,\\
			\hat{b} = 0.1386
		\end{cases} 
	\end{equation*}
	Коэффициент детерминации $$R_2^2 = 0.002.$$
	
	Сравнивая две нелинейных модели по коэффициенту детерминации, можно явно сказать, что первая нелинейная модель $y= a+b \cdot x^2 + \epsilon = 2.6562 + 1.3804 \cdot x^2 + \epsilon$  лучше, чем вторая.\\
	
	В то же время, сравнивая первую нелинейную модель с линейной моделью, основываясь на соответствующих коэффициентах детерминации, можно сказать, что первая нелинейная модель $y= a+b \cdot x^2 + \epsilon = 2.6562 + 1.3804 \cdot x^2 + \epsilon$ все же лучше, чем линейная, поскольку ее коэффициент детерминации $R_1^2 = 0.793$ значительно больше, чем коэффициент детерминации линейной модели $R_0^2 = 0.531$
\end{enumerate}
\end{document}