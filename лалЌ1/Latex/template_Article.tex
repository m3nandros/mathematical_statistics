\documentclass[12pt, letterpaper, twoside]{article}
\usepackage[T2A]{fontenc}
\usepackage[utf8]{inputenc}
\usepackage[russian,english]{babel}
\usepackage{amsmath,amssymb}
\usepackage{indentfirst}
\usepackage{hyperref}

\title{Описательная часть расчетной работы №1 по математической статистике за 6 триместр}
\author{Вихляев Егор, ММТ-2}
\date\today

\begin{document}

\maketitle

\section{Задание №1}

Проверить гипотезу случайности на 5$\%$-ном уровне значимости с помощью
критерия серий или инверсий.
\\ \\

$H_0$:  Результаты наблюдений представляют реализацию независимой повторной
выборки.\\ \\

\begin{enumerate} 
	\item Задаем уровень значимости  $\alpha$ = 0,05 единиц (5$\%$).
	\item Воспользуемся критерием инверсий (см. все вычисления в расчетной части).
	\item Поскольку, по результатам вычислений с помощью критерия инверсий, имеем неравенство $|T_{\text{набл}}|$ < $T_{\text{кр}}$, делаем вывод, что нулевая гипотеза $H_0$ принимается. Результаты наблюдений представляют реализацию независимой повторной выборки.
\end{enumerate}

\section{Задание №2}

Построить гистограмму относительных частот. Определить выборочные
характеристики: среднее, дисперсию, моду, медиану, асимметрию и эксцесс.
На основе визуального анализа гистограммы, а также выборочных
характеристик, выдвинуть гипотезу о виде закона распределения
исследуемого набора данных. Сделать выводы о свойствах гипотетического
распределения (наличие симметрии, близость к нормальному
распределению, близость среднего к медиане и т.д.).

\begin{enumerate} 
	\item Построить гистограмму относительных частот. \\
	Относительная частота определяется следующей формулой: $$w_i = \frac{n_i}{n},$$ где $n_i$ -- частоты, n -- количество вариант выборки.
	\begin{enumerate} 
		\item Для начала построим вариационный ряд.
		\item Находим минимальное значение $x_{min} = 0.5$.
		\item Находим максимальное значение $x_{max} = 4.84$.
		\item Находим размах вариации $R = x_{max} - x_{min} = 4.84 - 0.5 = 4.34$.
		\item Находим оптимальное количество интервалов $k = 1 + [log(2,n)] \\ (\text{где n -- объём выборки}) = 1 + [log(2,200)] = 8$.
		\item Находим длину интервала $h = \frac{R}{k} = 0.5425$.
		\item Далее строим таблицу, солержащую наши интервалы, среднее значение интервалов $x_i$, частоты $n_i$, относительные частоты $w_i$, плотности $\frac{w_i}{h}$.
		\item На основании данных таблицы, а именно первого столбца с интервалами и столбца с плотностями, строим гистограмму относительных частот.
	\end{enumerate}
	\item Определить выборочные характеристики: среднее, дисперсию, моду, медиану, асимметрию и эксцесс.
		\begin{enumerate} 
		\item Среднее (выборочное среднее) -- это среднее арифметическое всех значений выборки. В нашем случае, $\overline{x_{\text{в}}} = 1.51$.
		\item Дисперсия (выборочная дисперсия) -- это среднее арифметическое квадратов отклонений всех вариант выборки от её средней. В нашем случае, $S_{\text{в}}^2 = \frac{\sum_{i=1}^n(x_i-\overline{x_{\text{в}}})^2}{n} = 0.93$.
		\item Мода $M_0$ дискретного вариационного ряда -- это варианта с максимальной частотой. В нашем случае, $M_0 = 0.76$.
		\item Медиана $m_e$ вариационного ряда -- это значение, которое делит его на две равные части (по количеству вариант). В нашем случае, $m_e = 1.21$. 
		\item Ассиметрия --  характеризует меру скошенности полигона или гистограммы влево / вправо относительно самого высокого участка. В нашем случае, $A_3 = \frac{m_3}{\sigma_{\text{в}}^3} = 1.44 > 0$, то есть, распределение обладает существенной правосторонней асимметрией, что, хорошо видно по гистограмме.
		\item Эксцесс -- показатель остроты пика графика распределения. В нашем случае, $E_k = \frac{m_4}{\sigma_{\text{в}}^4} = 1.84 > 0$, то есть, распределение заметно выше, чем нормальное распределение с параметрами $\overline{x_{\text{в}}} = 1.51, \sigma_{\text{в}} = \sqrt{S_{\text{в}}^2} = 0.9643...$.
	\end{enumerate}
	\item На основе визуального анализа гистограммы, а также выборочных характеристик, выдвинуть гипотезу о виде закона распределения исследуемого набора данных. \\ \\
	На основе визуального аналига гистограммы, можно сделать вывод, что перед нами нечто похожее на показательный закон распределения, а именно на закон F4. Потому выдвинем следущую нулевую гипотезу.\\ \\ $H_0$: Исследуемый набор данных имеет вид закона распределения $$F4: f(x) = 2\lambda xe^{-\lambda x^2}, \text{при} \ x \geq 0, \lambda > 0$$.
	\item Сделать выводы о свойствах гипотетического распределения (наличие симметрии, близость к нормальному распределению, близость среднего к медиане и т.д.).
	\begin{enumerate} 
		\item Наличие симметрии. \\ \\
		Основываясь как на гистограмме гипотетического распределения, так и на его выборочных характеристиках, делаем вывод, что симметрия у распределения отсутствует. Гистограмма явно убывает по направлению оси ОХ, о чем говорит и ассиметрия $A_3$, согласно которой гистограмма значительно скошена вправо относительно самого высокого участка. Коэффициент эксцесса $E_k$ это подтверждает.
		\item Близость к нормальному распределению. \\ \\
		На основании гистограммы относительных частот и вычисленных выборочных характеристик, можно сделать вывод, что различие между исходным распределением и нормальным распределением статистически значимо и вряд ли объяснимо случайными факторами.
		\item Близость среднего к медиане. \\ \\
		Среднее $\overline{x_{\text{в}}} = 1.51$, медиана $m_e = 1.21$. Очевидно, разность между средним и медианой не столь велика, а потому они достаточно близки.
	\end{enumerate} 
\end{enumerate}

\section{Задание №3}

С помощью метода максимального правдоподобия и метода моментов
оценить неизвестные параметры гипотетического распределения. Построить
график плотности гипотетического распределения на том же рисунке, что и
гистограмма, используя вместо неизвестного параметра его статистическую
оценку (ОМП и ОММ).

\begin{enumerate} 
	\item Оценим неизвестные параметры гипотетического распределения методом максимального правдоподобия (ОМП). \\ \\
	Установим исходные данные. Имеем закон распределения $$F4: f(x) = 2\lambda xe^{-\lambda x^2}, \text{при} \ x \geq 0, \lambda > 0,$$ и, следовательно, один неизвестный параметр $\lambda$. \\
	\begin{enumerate} 
		\item Строим $L(\lambda;x_1,...,x_n) = \prod\limits_{i = 1}^n 2\lambda x_i e^{-\lambda x_i^2} I(x_i \geq 0) = (2\lambda)^n e^{-\lambda \sum_{i=1}^n x_i^2} \prod_{i=1}^n x_i  I(x_{(1)} \geq 0)$. 
		\item Логарифмируем $L(\lambda;x_1,...,x_n) = ln(L(\lambda;x_1,...,x_n)) = ln((2\lambda)^n e^{-\lambda \sum_{i=1}^n x_i^2} \prod_{i=1}^n x_i) = ln((2\lambda)^n) + ln(e^{-\lambda \sum_{i=1}^n x_i^2}) + ln(\prod_{i=1}^n x_i) = n*ln(2\lambda) - \lambda \sum_{i=1}^n x_i^2 + ln(\prod_{i=1}^n x_i)$.
		\item Строим уравнение $\frac{\partial ln L(\lambda;x_1,...,x_n)}{\partial \lambda}, \text{которое решаем относительно параметра} \ \lambda$.
		$$\frac{\partial ln L(\lambda;x_1,...,x_n)}{\partial \lambda} = \frac{\partial}{\partial \lambda} (n*ln(2\lambda) - \lambda \sum_{i=1}^n x_i^2 + ln(\prod_{i=1}^n x_i)) = \frac{n}{\lambda} - \sum_{i=1}^n x_i^2 = 0$$
		$$\widetilde{\lambda_{\text{ОМП}}} = \frac{n}{\sum_{i=1}^n x_i^2} = \frac{200}{640.7856} \approx 0.312$$

		\item Проверяем полученное значения на максимум: $$\frac{\partial^2}{\partial \lambda^2} = -\frac{n}{\lambda^2} < 0, \forall \lambda \Rightarrow$$
		$\Rightarrow$ найденное решение максимизирует $L(\lambda; x_1, ..., x_n)$ относительно параметров, найден максимум на границе области их изменения.
	\end{enumerate} 

	Таким образом, $\widetilde{\lambda_{\text{ОМП}}} \approx 0.312$.

	\item Оценим неизвестные параметры гипотетического распределения методом моментов (ОММ).\\ \\
	Итак, $M[X] = \overline{X}$. Выборочное среднее $\overline{X} = \overline{x_{\text{в}}} = 1.51$.\\
	$$M[X] = \int_0^{+\infty} 2\lambda x^2 e^{-\lambda x^2}dx = 2\lambda \int_0^{+\infty} x^2 e^{-\lambda x^2}dx = \frac{\sqrt{\pi}}{2\sqrt{\lambda}}$$
	$$\frac{\sqrt{\pi}}{2\sqrt{\lambda}} = 1.51$$
	$$\sqrt{\lambda} = \frac{\sqrt{\pi}}{3.02}$$
	$$\widetilde{\lambda_{\text{ОММ}}} = \frac{\pi}{9.1204} \approx 0.344.$$
\end{enumerate} 

\section{Задание №4}

С помощью критерия хи-квадрат проверить гипотезу о виде распределения с уровнем значимости $\alpha$, приведя все промежуточные расчеты. Вычислить $p$-значение критерия (реальный уровень значимости критерия).

$$H_0: X \sim 2\lambda xe^{-\lambda x^2} - \text{выборка подчиняется показательному закону}$$

Пусть уровень значимости $\alpha = 0.05$, число интервалов $k = 8$, число неизвестных параметров $r = 2$. $\widetilde{m_{\text{ОМП}}} = \overline{x} = 1.51, \widetilde{\sigma_{\text{ОМП}}^2} = S_{\text{в}}^2 = 0.93$.

$$\chi_{\text{крит}}^2 = x_{1-\alpha}[\chi_{k-r-1}^2] = x_{1-0.05}[\chi_{8-2-1}^2] = x_{0.95}[\chi_{5}^2] = 11.07.$$

Вычислим $\chi_{\text{выб}}^2 = \sum_{i=1}^k\frac{(n_i-n*p_i)^2}{n*p_i}$. Прежде всего, ищем вероятности $p_i$ по формуле $p_i = P(X_i \leqslant X \leqslant X_{i+1}) = \Phi(\frac{X_{I+1}-m}{\sigma_x}) - \Phi(\frac{X_I-m}{\sigma_x})$. Для упрощения расчетов, воспользуемся функцией ЭКСП.РАСП в Excel (все расчеты на 4 листе), подставляя в качестве параметра $\lambda = \widetilde{\lambda_{\text{ОМП}}} = 0.312$. Далее ищем значения $n*p_i$ для облегчения поиска итоговой суммы, а затем и элементы $\frac{(n_i-n*p_i)^2}{n*p_i}$ самой суммы. \\ 

Таким образом, $\chi_{\text{выб}}^2 = 105.12$. Отсюда ясно, что $$(\chi_{\text{выб}}^2 = 105.12) > (\chi_{\text{крит}}^2 = 11.07) \Rightarrow$$ $\Rightarrow$ нулевая гипотеза не принимается $\Rightarrow$ наша исходная выборка не подчиняется показательному закону.\\

\section{Задание №5}
Разделить набор данных на 2 части и, проверить гипотезу однородности этих частей. \\ \\

Итак, поделим исходную выборку пополам, получив, таким образом, две выборки X и Y равных размеров. Воспользуемся критерием Манна-Уитни, установим уровень значимости $\alpha = 0.05$ и выдвинем следующую нулевую гипотезу: $$H_0: F_x(x) = F_y(y).$$

Получаем, что $n = 100$ и $m = 100$ -- объёмы выборок X и Y соответственно. Ищем статистику $T_{\text{кр}}$: $$T_{\text{кр}} = x_{1-\frac{\alpha}{2}}[N(0;1)] = x_{0.975}[N(0;1)] = 1.96.$$

Чтобы найти $T_{\text{набл}} = \frac{|U - \frac{nm}{2}|}{\sqrt{\frac{nm(n+m+1)}{12}}}$, необходимо прежде найти вспомогательную функцию $U = \sum_{i=1}^n\sum_{j=1}^m I(x_i < y_j) + \frac{1}{2} \sum_{i=1}^n\sum_{j=1}^m I(x_i = y_j)$. По итогам расчетов (см. расчетную часть), $U = 5457$, тогда: $$T_{\text{набл}} = \frac{|5457 - \frac{100*100}{2}|}{\sqrt{\frac{100*100(100+100+1)}{12}}} = 1.117.$$

Таким образом, переходим к сравнению найденных статистик: $$(T_{\text{набл}} = 1.117) < (T_{\text{кр}} = 1.96) \Rightarrow$$ $\Rightarrow H_0$ -- принимается $\Rightarrow$ обе части X и Y из исходной выборки однородны.
\end{document}
